%%%%%%%%%%%%%%%%%%%%%%%%%%%%%%%%%%%%%%%%%
% Beamer Presentation
% LaTeX Template
% Version 1.0 (10/11/12)
%
% This template has been downloaded from:
% http://www.LaTeXTemplates.com
%
% License:
% CC BY-NC-SA 3.0 (http://creativecommons.org/licenses/by-nc-sa/3.0/)
%
%%%%%%%%%%%%%%%%%%%%%%%%%%%%%%%%%%%%%%%%%

%----------------------------------------------------------------------------------------
%	PACKAGES AND THEMES
%----------------------------------------------------------------------------------------

\documentclass{beamer}

\mode<presentation> {

% The Beamer class comes with a number of default slide themes
% which change the colors and layouts of slides. Below this is a list
% of all the themes, uncomment each in turn to see what they look like.

%\usetheme{default}
%\usetheme{AnnArbor}
%\usetheme{Antibes}
%\usetheme{Bergen}
%\usetheme{Berkeley}
%\usetheme{Berlin}
%\usetheme{Boadilla}
\usetheme{CambridgeUS}
%\usetheme{Copenhagen}
%\usetheme{Darmstadt}
%\usetheme{Dresden}
%\usetheme{Frankfurt}
%\usetheme{Goettingen}
%\usetheme{Hannover}
%\usetheme{Ilmenau}
%\usetheme{JuanLesPins}
%\usetheme{Luebeck}
%usetheme{Madrid}
%\usetheme{Malmoe}
%\usetheme{Marburg}
%\usetheme{Montpellier}
%\usetheme{PaloAlto}
%\usetheme{Pittsburgh}
%\usetheme{Rochester}
%\usetheme{Singapore}
%\usetheme{Szeged}
%\usetheme{Warsaw}

% As well as themes, the Beamer class has a number of color themes
% for any slide theme. Uncomment each of these in turn to see how it
% changes the colors of your current slide theme.

%\usecolortheme{albatross}
%usecolortheme{beaver}
%\usecolortheme{beetle}
%\usecolortheme{crane}
%\usecolortheme{dolphin}
%\usecolortheme{dove}
%\usecolortheme{fly}
%\usecolortheme{lily}
%usecolortheme{orchid}
%\usecolortheme{rose}
%\usecolortheme{seagull}
\usecolortheme{seahorse}
%\usecolortheme{whale}
%\usecolortheme{wolverine}

%\setbeamertemplate{footline} % To remove the footer line in all slides uncomment this line
%\setbeamertemplate{footline}[page number] % To replace the footer line in all slides with a simple slide count uncomment this line

\setbeamertemplate{navigation symbols}{} % To remove the navigation symbols from the bottom of all slides uncomment this line
}
%\usepackage[brazilian]{babel}
\usepackage[utf8]{inputenc}
\usepackage{pgfplots}
\usepackage{graphicx} % Allows including images
\usepackage{booktabs} % Allows the use of \toprule, \midrule and \bottomrule in tables
\usepackage{courier}
%----------------------------------------------------------------------------------------
%	TITLE PAGE
%----------------------------------------------------------------------------------------

\title[EP3]{EP3 de MAC0422} % The short title appears at the bottom of every slide, the full title is only on the title page

\author{Gabriel Capella (8962078) e Luís Felipe de Melo Costa Silva (9297961)} % Your name
\institute[USP] % Your institution as it will appear on the bottom of every slide, may be shorthand to save space
{
IME-USP\\ % Your institution for the title page
\medskip
}

\begin{document}

\begin{frame}
\titlepage % Print the title page as the first slide
\end{frame}

%----------------------------------------------------------------------------------------
%	PRESENTATION SLIDES
%----------------------------------------------------------------------------------------

%------------------------------------------------
%\section{Problema} % Sections can be created in order to organize your presentation into discrete blocks, all sections and subsections are automatically printed in the table of contents as an overview of the talk
%------------------------------------------------

%\subsection{Subsection Example} % A subsection can be created just before a set of slides with a common theme to further break down your presentation into chunks

\begin{frame}
	\frametitle{Problema}
	\begin{itemize}
		\item Implementar um simulador de gerência de memória com diversos algoritmos para gerência do espaço livre e para substituição de páginas.
		\item Os algoritmos de gerência de espaço livre são:
		\begin{enumerate}
			\item First Fit
			\item Next Fit
			\item Best Fit
			\item Worst Fit
		\end{enumerate}
		\item Os algoritmos de substituição de página são:
		\begin{enumerate}
			\item Optimal
			\item Second-Chance
			\item Clock
			\item LRU (Quarta Versão)
		\end{enumerate}
	\end{itemize}
\end{frame}

\begin{frame}
	\frametitle{Implementação}
	\begin{itemize}
		\item A linguagem escolhida foi Python, que é orientada a objetos e de script, sendo, portanto, multiparadigma.
		\item Isso facilitou bastante a leitura do arquivo de trace, no armazenamento das informações, na alocação e desalocação da memória utilizada na execução do programa e no tratamento dos objetos durante os trabalhos.
	\end{itemize}
\end{frame}

%------------------------------------------------
\begin{frame}
	\frametitle{Módulos}
	\framesubtitle{ep3.py}
	\begin{itemize}
		\item Implementa o console no terminal, a interação com o usuário.
		\item É um envelope. Apenas lê as informações e então chama as funções responsáveis pelo comando lido.
		\item Os comandos são os que foram definidos pelo enunciado (\texttt{carrega}, \texttt{espaco}, \texttt{substitui}, \texttt{executa} e \texttt{sai}), com os respectivos argumentos.
		\item Utiliza a biblioteca \texttt{cmd} do Python.
		\item Depende do arquivo gerenciador.py
	\end{itemize}
\end{frame}

\begin{frame}
	\frametitle{Módulos}
	\framesubtitle{gerenciador.py}
	\begin{itemize}
		\item Faz o trabalho pesado pedido pelo módulo anterior (possui as funções que ele chama).
		\item Possui as seguintes funções:
		\begin{itemize}
			\item Leitura do arquivo de trace.
			\item Escolha do algoritmo de gerência de espaço livre e do algoritmo de substituição de página.
			\item Execução do simulador com as informações recebidas.
		\end{itemize}
		\item Depende dos módulos proccess.py e substitui.py
	\end{itemize}
\end{frame}

\begin{frame}
	\frametitle{Módulos}
	\framesubtitle{substitui.py}
	\begin{itemize}
		\item Possui duas classes: \texttt{Pagina} e \texttt{Substitui}.
		\item A classe \texttt{Pagina} inicializa o objeto \texttt{Pagina} com as seguintes informações: qual memória ela está, o número do processo, o número que o processo enxerga, o bit R, a próxima referência (para o Optimal), onde ele está na memória e quando ela foi chamada pela última vez.
		\item A classe \texttt{Substitui} possui as seguintes funções:
		\begin{itemize}
			\item Impressão de informações no console.
			\item Atribuição das informações lidas no console.
			\item Função que tira um processo pronto da fila.
			\item Configuração do bit R.
			\item Page fault (tem os algoritmos de substituição).
			\item Acesso à "memória."
		\end{itemize}
		\item Depende dos arquivos espaco.py, process.py e gerenciador\textunderscore arquivo.py
	\end{itemize}
\end{frame}

\begin{frame}
	\frametitle{Módulos}
	\framesubtitle{process.py}
	\begin{itemize}
		\item Possui a definição do objeto \texttt{process}, que armazena as seguintes informações:
		\begin{itemize}
			\item PID
			\item Nome do processo
			\item Início e fim
			\item Páginas que ele acessa
			\item Quando acessa cada página.
		\end{itemize}
	\end{itemize}
\end{frame}

\begin{frame}
	\frametitle{Módulos}
	\framesubtitle{gerenciador\textunderscore arquivo.py}
	\begin{itemize}
		\item Esse arquivo cuida de escrever nos arquivos \texttt{/tmp/ep3.men} e \texttt{/tmp/ep3.vir}. Caso não existam, são criados.
		\item Possui as funções:
		\begin{itemize}
			\item Cria os arquivos e os enche com $-1$.
			\item Escreve em \texttt{/tmp/ep3.men}.
			\item Escreve em \texttt{/tmp/ep3.vir}
		\end{itemize}
	\end{itemize}
\end{frame}

\begin{frame}
	\frametitle{Módulos}
	\framesubtitle{espaco.py}
	\begin{itemize}
		\item Esse módulo tem os algoritmos de gerência de memória implementados.
		\item Basicamente, possui duas funções:
		\begin{itemize}
			\item Devolver um endereço de espaço (usando o algoritmo escolhido).
			\item Liberar um espaço de memória que não será maus usado.
		\end{itemize}
	\end{itemize}
\end{frame}

\begin{frame}
	\frametitle{Entrada}
	\begin{itemize}
		\item O módulo do console trabalha com essa parte.
		\item Um exemplo está abaixo:
		\begin{figure}[!h]
			\centering
			\includegraphics[scale=0.75]{a.png}
		\end{figure}
		\item Como podemos ver, o arquivo trace será lido, o algoritmo de substituição de página utilizado será o Optimal, o de gerenciamento de espaço livre será o Worst Fit e a impressão da execução será de 10 em 10 "segundos".
	\end{itemize}
\end{frame}

\begin{frame}
	\frametitle{Saída}
	\begin{itemize}
		\item Para a entrada anterior, temos a seguinte saída:
		\begin{figure}[!h]
			\centering
			\includegraphics[scale=0.5]{b.png}
		\end{figure}		
	\end{itemize}
\end{frame}

\begin{frame}
	\frametitle{Saída}
	\begin{itemize}
		\item No arquivo de trace foi definido que a memória física teria tamanho 64, a virtual, 1024, e que o tamanho das páginas (s e p) seria 4.
		\item Podemos ver então, no bitmap da memória física, 16 bits (64/4) e no da virtual, 256 (1024/4).
		\item 0 é para posição livre e 1 é para posição ocupada.
	\end{itemize}
\end{frame}

\end{document} 